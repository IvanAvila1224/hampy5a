\documentclass{purdue-slide}

% For filler text:
\usepackage[base]{babel}
\usepackage{graphics}
\usepackage{lipsum}


\title {\tiny }
\subtitle{\tiny Made with Jaime Ivan Avila Mu˜noz}
\author{}
\institute{Analysis and documentation of practice 5A:
Numerical integration with Simpson’s rule and
the t distribution}
\date{\today}
\renewcommand{\slidefoot}{\LaTeX\ Template}

\begin{document}

\begin{titleframe}{}
    \maketitle
\end{titleframe}

\begin{titleframe}{\tiny 1.1 Objectives
Calculation of the numerical integration applying Simpson’s rule, together with
the T distribution, checking the results with the tests suggested in the ”issues”
located in the Hampty5A repository in github.}

{\tiny
Numerical integration with Simpson’s rule

Numerical integration is the process of determining the area
“under” some function.
Numerical integration calculates this area by dividing it into vertical
“strips” and summing their individual areas.
The key is to minimize the error in this approximation.}

\includegraphics[width=2 cm, height= 2cm]{image.png}
   
\end{titleframe}

\section{Long Text}

\begin{frame}{\tiny Rule Simpson}
   \tiny Simpson’s rule
\\1. num_seg = initial number of segments, an even number

\\2. W = x/num_seg, the segment width
\\3. E = the acceptable error, e.g., 0.00001
\\4. Compute the integral value with the following equation.
\\5. Compute the integral value again, but this time with num_seg = num_seg*2.
\\6. If the difference between these two results is greater than E, double
\\num_seg and compute the integral value again. Continue doing this until
\\the difference between the last two results is less than E. The latest result is
the answer.
TEST

\\Should return p=16 if f(x)=2x, x0=0, x1=4, num_seg=4, dof=0
\\Should return p=0.33333 if f(x)= x*x, x0=0, x1=1, num_seg=4, dof=0
\\Should return p=1.386 if f(x)= 1/x, x0=1, x1=4, num_seg=6, dof=0

    \vspace{\baselineskip}
    
   
\end{frame}

\section{List}

\begin{frame}{T distribution}
   \tiny The t distribution
\\The t distribution is a very important statistical tool. It is used instead of \\the normal distribution when the true value of the population variance is not \\known and must be estimated from a sample.
\\The shape of the t distribution is dependent on the number of points in your \\dataset. As n gets large, the t distribution approaches the normal distribution.
\\For lower values, it has a lower central “hump” and fatter “tails.”
\\The t distribution is a very important statistical tool. It is used instead of \\the normal distribution when the true value of the population variance is not \\known and must be estimated from a sample.
\\The shape of the t distribution is dependent on the number of points in your \\dataset. As n gets large, the t distribution approaches the normal distribution.
\\For lower values, it has a lower central “hump” and fatter “tails.”
    \begin{itemize}
       
    \end{itemize}
\end{frame}

\begin{frame}{}
    \tiny Some introduction of the list.
    Test
\\1. First we’ll set num_seg = 10 (any even number)
\\2. W = x/num_seg = 1.1/10 = 0.11  (any even number)
\\3. E = 0.00001  (any even number)
\\4. dof = 9  (any even number)
\\5. x = 1.1  (any even number)
\\6. Compute the integral value with the following equation.
\\7. We can solve the first part of the equation:
\\Compute the integral value again, but this time with num_seg = 20. The
\\new result is 0.35005864.
\\8. We compare the new result to the old result.
\\9. 0.3500589−0.35005864 < E
\\10. We can then return the value p =0.35005864.
\\TEST

\\Should return P=0.35006 if f(x)x0=0, x1=1.1, num_seg=10, dof=9
 
\\Should return P=0.33333 if f(x)x0=0, x1=1.1812, num_seg=10, dof=10
 
\\Should return p=1.386 if f(x)x0=0, x1=2.750, num_seg=10, dof=30
    
\end{frame}



\end{document}